% Example LaTeX document for GP111 - note % sign indicates a comment

\documentclass[12pt]{article}
% Default margins are too wide all the way around.  I reset them here
\setlength{\topmargin}{-.5in}
\setlength{\textheight}{9in}
\setlength{\oddsidemargin}{.125in}
\setlength{\textwidth}{6.25in}
\usepackage{setspace}
\usepackage[pdftex]{graphicx}
\begin{document}
\title{Browser-based Categorization of Data towards Automated Visualization}

\maketitle







\section*{Civic Worker}
Tom works for Fingal County Council. He has been tasked with drafting new proposals to help combat unemployment. Part of his work involves identifying the areas of highest unemployment within Fingal. He does this my dividing the number of unemployed people in an area by the population of that area. To illustrate his findings he inputs the initial dataset into the system described and is presented with a number of visualizations. These findings are then used to target the new proposals to where they are needed most.

\section*{CFO of coffee shop chain}  
The chief financial officer (CFO) of a coffee store chain has been told to cut expenses. This will involve a detailed analysis of income and expenses for every store and every product sold. To get an initial understanding of the situation the CFO visualizes the relationship between marketing costs and profit categorized by product type (and maybe market, not sure yet). The CFO inputs the data into the system presented and is presented with a number of visualizations. After studying the graphics the CFO notices certain products have high marketing costs with little to no return.

The CFO continues his analysis, investigating the sales performance of this product by region. He inputs the marketing cost and profit  for every region as again presented with a series of visualizations. He notices that the product does not perform badly in every region, but in some regions sales are almost non-existent. With this data the CFO can change the company's marketing and sales strategies in the relevant regions.

\section*{Third use case}
Something involving a relatively novel chart.



\end{document}
